\chapter{Conclusion and Future Direction}
In this thesis, a novel single-cell whole genome sequencing technology termed \textit{virtual microfluidics} was developed and applied to enable the study of genomic heterogeneity in complex biological systems. Our technology establishes a new paradigm in single-molecule and single-cell analysis with dramatically different characteristics than established microfluidic approaches. Applications of the technology on purified DNA, cultured bacteria, human gut microbiome samples, and human cell lines demonstrated the robustness of the system. Below, the key findings of this thesis are summarized and possible future research directions are proposed.

\section{Summary of advancements}
\textit{Virtual microfluidics} enables high-throughput nucleic acid digital quantification and whole genome amplification in an easy-to-use, benchtop format that requires no special equipment or environmental control. 

% Besides reducing the production of chimeras in MDA, the unique physical characteristics of the engineered hydrogel environment provide a means for enhancing coverage extent and uniformity from WGA and WTA samples through the self-limiting reactivity within each virtual compartment. In addition, the straightforward addition and removal of reagents to\slash from product clusters \textit{en masse} and excellent optical access ideally suit the \textit{virtual microfluidics} system for rare-cell assays incorporating \textit{in situ} labeling of cells or product clusters. We expect that \textit{Virtual microfluidics} will find application as a low-cost digital assay platform and as a high-throughput platform for single-cell sample preparation. 

\subsection{Enabling the equipment-independent high-throughput DNA target detection}
In chapter 2, we demonstrated \textit{virtual microfluidics} as a robust nucleic acid quantification platform. The dynamic range of the measurement from a tradition method (such as dPCR in droplets or microfluidics devices) for DNA quantification is restricted by the number of partitions, usually up to $10^5$. Due to the nature of \textit{virtual microfluidics}' diffusion-restricted reaction and the continuous virtual chambers, up to 20,000,000 analytes per $\mu$L could be accommodated in our system. Specifically, we tested the performance of in-gel digital PCR and digital MDA as an analytical method for molecular detection and counting. We demonstrated high-throughput digital assays and preparative whole-genome amplification without microfabricated consumables or expensive instrumentation. As few as one DNA target can be detected microscopically with a high signal-to-noise ratio by DNA amplification. The in-gel amplification environment also seals potential infectious targets to minimize the handling of biohazardous materials for infectious disease diagnosis. We expect that \textit{virtual microfluidics} will find application as a low-cost digital assay for detecting DNA biomarkers in the clinic. 

\subsection{Improving the whole genome sequencing data quality and success rate for characterizing uncultured microorganisms}
In chapter 3, we characterized whole genome amplification and recovery of single bacterial genomes for lab-cultured control cells and the human gut microbiome using next-generation sequencing (NGS). Compared to traditional methods of whole genome sequencing on single microbes (in tube and microfluidic device), we improved the uniformity of the whole genome amplification by 25\% $\sim$ 33\% and reduced the rate of the chimeric artifact by a factor of six. The success rate of \textit{virtual microfluidics} single-cell sequencing is about 28\%, which is limited by the Poisson distribution. In contrast, typical success rates (the percentage of amplified genomes that pass purity and genome-size threshold) of single-cell sequencing services provided by large-scale genomic centers based on the first-hand experience from our collaborators is about 10\%. Such genomic centers routinely conduct single-cell sequencing in the clean room and utilize FACS for cell isolation, while our approach has a minimal engineering requirement. We demonstrated single-cell sequencing on human gut microbiome samples and obtained 117 pure single draft genomes. Working with collaborators, we were able to utilize the draft genomes to identify more than 10,000 horizontally transferred genes with unique population-specific and individual-specific features \cite{Brito:2016cd}. We expect that \textit{virtual microfluidics} will find application as a high-throughput platform for single-cell sample preparation to study a diverse collection of uncharacterized microbes and environmental microbiome samples. 

\subsection{Reducing structural variation artifacts for studying human cells using single-cell sequencing}
In chapter 4, we demonstrated \textit{virtual microfluidics} for high-quality single-cell genome sequencing on human cell lines with a 3 $\sim$ 6 fold chimera artifact reduction compared to several single-cell technologies. The chimera reduction feature of the \textit{virtual microfluidics} reduces the false-positive rate of genome structural variation detection in studying tumor clonal heterogeneity. Bioinformatically, we characterized chimeric DNA rearrangements in several recently developed single-cell technologies. The unique chimera signatures across different platforms drew attention to the importance of characterizing chimera artifacts in newly developed single-cell technologies. The hydrogel environment also eliminates the need of creating ultra-small discrete chambers for sub-microliter MDA reactions for comparable high-quality data. Furthermore, all of the preparative steps of single-cell whole genome sequencing are accessible with basic lab equipment. We expect \textit{virtual microfluidics} to be implemented widely by researchers who are interested in studying tumor heterogeneity with single-cell resolution.  
% The digital MDA aspect enables real-time monitoring of the whole genome amplification reaction and could be utilized to determine the fragmentation level of the genome and the extent of amplification.

To summarize, this thesis work centers on the development and demonstration of \textit{virtual microfluidics}, a novel technique for high-quality low-input genomic research. This technique makes single-cell genomics more accessible to a wide range of scientific and biomedical researchers. 

\section{Future direction}
% \subsection{Future applications}
% \subsubsection{}
% \subsubsection{}
% For environmental microbiologists who samples uncharacterized bacteria in the backcountry and on the research vessel
% We expect \textit{virtual microfluidics} find applications as a low-cost, highly accessible digital assay platform that provides superior sensitivity and dynamic range.
% For applications that are sensitive to DNA contaminants, our benchtop technology performance can be further improved by using extra measures against contamination, such as the use of a laminar flow cabinet, separate pre\slash post-amplification laboratories and equipment
\subsection{Future technical improvements of \textit{virtual microfluidics}}
Additional technical improvements are needed in order to realize the full potential of \textit{virtual microfluidics}. To begin, the throughput of \textit{virtual microfluidics} can be improved further. Currently, the primary throughput limitation in the initial demonstration on microbial samples is the volume sub-sampled (60 nL) when product clusters are retrieved, which limits the number of sub-samples that can be retrieved from a single hydrogel. A number of approaches are worth exploring: using a thinner gel with more surface area and\slash or reducing the punch size from the 500 $\mu$ m and 1 mm diameters we employed here could improve throughput. A second possibility could be to use imaging data to guide product retrieval and increase the fraction of retrieved samples containing a single-cell WGA reaction product. The thin hydrogel format affords excellent physical access for imaging, equipment, and reagents, which enables an assortment of sub-sampling approaches including punch\slash pickers, localized hydrogel dissolution, and localized affinity tagging or barcoding. Finally, barcoding approaches could conceivably enable retrieval of all amplified products \textit{en masse} while allowing \textit{in silico} demultiplexing to sort sequence reads according to the cell of origin \cite{Crosetto:2015vd}.

\subsubsection{Suitability of in-gel amplification format for product cluster labeling}
\textit{Virtual microfluidics}'s excellent optical accessibility allows potential fluorescent labeling of rare sequences, which is essential to identify rare targets in microbial dark matter discovery and liquid biopsy applications. For these applications, the demand for single-cell assay throughput is not driven by the need to amass a large number of single-cell datasets, but rather to access cells that are rare in the population. The hydrogel format is ideally suited for this case as the WGA reaction endpoint is an opportune moment to genotype product clusters using hybridization probes in order to identify cells of interest for retrieval and sequencing analysis \cite{Niki:1997vq,Yamada:2011kf}. In the post-reaction hydrogel, genomic sequences have been amplified and are not protected by a cell envelope. In addition, the thin gel slab format facilitates the application of reagents for rapid template denaturation, labeling, and de-staining. Once labeled, the desired targets can be selectively retrieved for further analysis by image-guided selection. Sequence-specific labeling might also reduce the number of false-positive background spots that challenged the intercalating dye-based approach we used in this study and \slash or to lend molecular specificity to quantification assays. In fact, sequential FISH could be used to probe for large sets of functional genes within the gel itself, enabling the application of complex selection criteria \cite{Lubeck:2014jx}. 

\subsubsection{Potential for amplification bias reduction}
Up to 10 pg of DNA is produced by MDA from each template in the hydrogel format using our protocol. Although we re-amplified punch samples in the microbial study to microgram quantities, 10 pg is, in principle, enough product (of an order 1000 bacterial genome equivalents) to support deep sequencing directly. Given that we obtain good coverage distribution with our high-yield re-amplification protocol for bacteria, it may be possible for coverage distribution improvement by direct library construction from the 10 pg hydrogel product. Recent advances in ultra-efficient library construction have demonstrated library construction from sub-nanogram input levels \cite{White:2009jz}. 

Although a number of modified protocols have been proposed to improve coverage distribution in MDA, none has yet been widely adopted, with major single-cell genomics centers continuing to use $\Phi$29 DNA polymerase reaction conditions very similar to those originally developed 30 years ago \cite{Blanco:1985ul,Allen:2011jn}. In contrast, limiting fold-amplification reduces coverage bias, since the ratio of maximum possible fold amplification to minimum possible fold amplification is necessarily reduced when the average degree of amplification is reduced. When combined with the cost savings of micro-scaled reactions and increasingly efficient sequence library construction procedures, such an approach shows the future trend of single-cell WGA \cite{deBourcy:2014ji}. 

Today, investigators limit amplification-fold by reducing reaction volumes \cite{Marcy:2007ip,Landry:2013dh} or by limiting reaction time \cite{Spits:2006em}. Although it is currently unknown which approach is more fruitful in bias reduction, both approaches have drawbacks. The hydrogel reaction format offers unique advantages in limited-extent WGA, as the product clusters from each template molecule only reach a few microns in size, even under dilute template conditions. This suggests that one can achieve uniform (limited) reaction extent across single-cell WGA reactions, even when the reactions occur asynchronously. The hydrogel format also enables maintenance of optimal amplification conditions for each template throughout the reaction time course if desired by reagent supplementation, possibly reducing sequence content and template fragment length biases. In order to further measure the amplification bias in the hydrogel, a random barcode library can be introduced into the hydrogel for amplification and sequencing analysis. 

\subsection{The future of single-cell whole genome sequencing}
The single-cell sequencing process from having cells in suspension to obtaining sequencing data is highly fragmented in terms of technology implementation. On the other hand, due to the diverse applications of single-molecule and single-cell analysis, it is difficult to find a one-tool-fits-all solution. The key is to have a platform technology that allows modular changes of different processes involved, in order to strike a balance among the requirements of throughput, hands-on time, cost, and quality for various applications. This reality also explains the slow uptake of single-cell technology in various research and clinical settings. To address these issues, \textit{virtual microfluidics} represents a platform technology that can be implemented with a diverse collection of methods in cell lysis, whole genome amplification, and barcoding strategies. With further technical optimizations, this technique could play a central role in integrating the single-cell sequencing field.

% Beyond the traditional method of whole genome amplification on isolated single cells, I envision that it will be ideal to barcode a large number of single cells with minimal amplification and obtain accurate long-read sequencing results. Such a technology combination will revolutionize the landscape of single-cell sequencing. Because it minimizes the biases and artifacts from extensive amplification that often distort the output data from the original genomic sequences. Long-read sequencing (currently 10 kbp - 400 kbp) has the advantage of providing the genuine read of a long genome sequence that could be highly repeated and of high GC\%, which are challenging for the current short-read technologies (50 bp - 500 bp). At the same time, it also requires the development of more accurate sequencing technologies. Currently PacBio (long-read) has an error rate of $\sim 10\%$, compared to the 0.1 $\sim$ 1 \% of the short-read Illumina sequencing. A recent development has demonstrated direct library preparation on single cells in nano-droplets for whole genome sequencing \cite{Zahn:2017fb}. Its performance is below that of state of the art single-cell technologies, in terms of genome physical coverage and coverage uniformity. However, the concept of direct library preparation on a large number of samples (hundreds) without the expensive and often biased WGA process is attractive to researchers to obtain the most representative genomic information. 

% Beyond single-cell sequencing, there still exist challenges in implementing genomic testing in the clinic as the standard of care. First of all, the reimbursement and clinical adoption reply on the innovation of the sequencing cost reduction. Secondly, it is difficult for patients, doctors and researchers who lack the genomic related training to interpret the data. This is especially critical when doctors and patients have to make decisions with the consideration of the inherent false-positive and false-negative rate of the genomic data. 
% Furthermore, a large percentage (40$\%$ - 60$\%$) of patients often don't have actionable mutations according to recent cancer sequencing projects (GenomeWeb Mar 01, 2017). The uptake of standard genomic technologies will likely develop hand-in-hand with genomic-guided and targeted drug discovery. With the growing of related products such as 23andMe, and genomic service companies such as Foundation Medicine, Grail, and Color Genomics, the share of genomic data to promote research and diagnosis will spread rapidly. In my opinion, the future of single-cell sequencing field relies on the uptake of standard genomic technologies, in addition to more single-cell analysis innovations, in order to realize its full potential.  
% % I feel like it would be nice to fit your work a little more closely with your conclusion somehow.
% % \subsubsection{The origin story of \textit{virtual microfluidics}}
% % In 2012, I started testing the DNA amplification in the hydrogel with a project code name \say{matrix digital}. The goal was to achieve a sub-cellular resolution for \textit{in-situ} RNA molecule counting. However, optimizing the hydrogel for 


